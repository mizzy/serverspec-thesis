%%
%% ��������ѥ����å�
%% [techrep]
%%
%% ��ʸ�ѥ����å�(keyword��Ǥ��)
%% [english]
%%

\documentclass[techreq,english]{ipsj}



\usepackage[dvips]{graphicx}
\usepackage{latexsym}

\def\Underline{\setbox0\hbox\bgroup\let\\\endUnderline}
\def\endUnderline{\vphantom{y}\egroup\smash{\underline{\box0}}\\}
\def\|{\verb|}

\setcounter{volume}{21}% vol21=2013
\setcounter{number}{1}
\setcounter{page}{1}

%\received{2011}{7}{1}
%\rereceived{2011}{10}{1}   % optional
%\rerereceived{2011}{10}{31} % optional
%\accepted{2011}{11}{5}

\usepackage[varg]{txfonts}%%!!
\makeatletter%
\input{ot1txtt.fd}
\makeatother%

\begin{document}

\title{How to Prepare Your Papers for the JIP}

\affiliate{IPSJ}{Information Processing Society of Japan, 
Chiyoda, Tokyo 101--0062, Japan}
\affiliate{JU}{Johoshori University, Chiyoda, Tokyo 101--0062, Japan}
\paffiliate{PJU}{Johoshori University}

\author{Joho Taro}{IPSJ,PJU}[joho.taro@ipsj.or.jp]
\author{Shori Hanako}{JU}[shori.hanako@johosyori-u.ac.jp]
\author{Gakkai Jiro}{JU}

\begin{abstract}
This document is a guide for preparing drafts to be submitted to the
Journal of Information Processing (JIP) and for the final camera-ready
manuscripts of papers to appear in the JIP that use \LaTeX and special
style files.  Since this document itself is produced with these style
files, it will help you to refer to its source file, which is
distributed with these style files.
\end{abstract}

%\begin{keyword}
%Journal of Information Processing, \LaTeX, style files, ``Dos and
% Don'ts'' list
%\end{keyword}

\maketitle

%1
\section{Introduction}

The Information Processing Society of Japan (IPSJ) publishes Journal of
Information Processing (JIP) as its flagship international journal.
Thus far, the JIP adopted the landscape A4 format for publishing papers,
but it has changed this format to the portrait A4 format because of many
requests from authors.  Corresponding to this format change, the JIP
accepts the portrait A4 format for submitting papers.

Following with this change, we, the Editorial Board of the JIP, prepared
a new style file for \LaTeX\@.  In this manuscript, we first describe
the usage of the style file.  The basic strategy of the new style file
is to require no special knowledge of command usage that does not use
standard \LaTeX commands.  Authors of a paper can use standard \LaTeX
commands to keep within the formatting restrictions of the paper, such
as setting space pitches and margins.  The guideline of the paper format
will be described in Section~\ref{body}.  Since this manuscript itself
also is written with the style file, we hope it will help with writing a
paper.

The Editorial Board of the JIP has also prepared a ``Dos and Don'ts''
list of matters an author should consider while writing a paper.  We
have added the list to the latter portion of this manuscript.  Please
use the list as a checklist for preparing to submit a paper.

%2
\section{Flow from Submission to Publishing}
%2.1
\subsection{Preparation}

The JIP author's kit including the \LaTeX style files can be downloaded
from the following URL:
\begin{quote}
 \small
 \|http://www.ipsj.or.jp/jip/submit/style.html|
\end{quote}
The kit includes the following files:
\begin{enumerate}%{
\item \|ipsj.cls|: style file for ipsj journals
\item \|ipsjdraft.sty|: style for drafts to be submitted
\item \|ipsjpref.sty|: style for the foreword
\item \|jsample.tex|: source for the Japanese version of this guide
\item \|esample.tex|: source of this guide
\item \|ipsjsort-e.bst|: bibtex style (sorted)
\item \|ipsjunsrt-e.bst|: bibtex style (unsorted)
\item \|bibsample.bib|: sample of bibliographic data (Japanese)
\item \|ebibsample.bib|: sample of bibliographic data (English)
\end{enumerate}%}

Since the kit has variants corresponding to multiple platforms,
including UNIX workstations, Windows (DOS), and Macintosh machines, an
appropriate variant can be selected and unpacked on the target
platform. 

Since {\LaTeXe} is required as an execution environment, please install
it.

Regarding manuscripts written with Microsoft Word, a corresponding
company will convert them into \LaTeX\@.  This means that the Microsoft
Word format is used just a reference.


\footnotetext{The real author is the Editorial Board of JIP.}

%2.2
\subsection{Draft Submission}

First, generate a PDF file from your \LaTeX source and style file under
your {\LaTeXe} environment and check that the generated PDF file can be
read with the Adobe PDF reader.  After that, register your email address
into the Paper Review Management System (PRMS) through the following
URL:
\begin{quote}
 \small
 \|https://www.ipsj.or.jp/prms/author_pre_submit.do|
\end{quote}

\noindent
The system will return an email including another URL for submitting
your paper. The manual for submission via the PRMS is available at the
following URL: 
\begin{quote}
 \small
 \|http://www.ipsj.or.jp/jip/submit/manual/|\\
 \|e_manual.html|
\end{quote}

The JIP adopts \textit{double blind review}, where reviewers of your
paper will not know your name, and you will not know theirs.  To ensure
that this is possible, the submitted draft version should not contain
information about the authors.

%2.3
\subsection{Final Version}

After you receive the notification of acceptance, revise your paper in
accordance with the comments from the referees and add the required
omissions from the draft, such as a biography, if any. The layout of
figures and tables should be fixed. After that, \textit{check your paper
again and again to completely remove description errors}.

Send \textit{both the {\LaTeX} file package and the hard copy} to the
IPSJ\@. The standard contents of the file package are .tex and .bbl. If
you include PostScript files and/or special style files, add them into
the package. Note that \textit{you must not split your source into
multiple .tex files} because it is hard for printers to access multiple
files when they modify your source. Also, carefully make sure that the
package contains all necessary files, especially special style files.

Details on the file transfer, including its destination and packaging
method, will be provided to you by the IPSJ secretariat.

%2.4
\subsection{Proofreading, Typesetting, and Publishing}

The IPSJ may change terms in your paper as per its standard, and the
printing house may modify your source to make it fit the standard
printing style.  Even if they make no changes, the result printed at the
printing house may be different from what you printed because of
differences in the {\LaTeX} execution environment.  Therefore, the
galley proofs of your paper will be sent to you so that you can check if
those modifications and/or differences are acceptable.  If not, correct
errors with red ink.  Note that \textit{this proofreading is not for
correcting your errors}, which should have been corrected before sending
the final version.

Your paper will be typeset after errors you notify us about (if any) are
corrected and will be published as part of the JIP.

%3
\section{Guide for Formatting a Paper}

The JIP, as opposed to conference proceedings, has a traditional and
\textit{stiff} style.  This makes the style files also \textit{stiff}
and strongly restricts customizability, which is one of the most useful
features of {\LaTeX}.  For example, you must not change \textit{style
parameters}, such as \verb+\texheight+.  It is not easy to show which
customizations are allowed, but the rule ``Don't tamper with it unless
you are confident'' should suffice.

Note that if you do something you should not, \textit{you may not have
error messages but simply unattractive results}.

The source file must use the following format.  Underlined parts can be
omitted from draft versions.

%4
\section{Configuration of Paper}
\label{body}

The source file must use the following format.  Underlined parts can be
omitted from draft versions.  Note that a few additional commands, shown
in A.1 of the Appendix, are available for a paper included in the
Transactions.

\vskip\baselineskip

\noindent
\|\documentclass[JIP]{ipsj}|\ or\\
\|\documentclass[JIP,draft]{ipsj}|\\
\quad Specify other option styles if necessary.\\
\quad Specify auxiliary styles with \|\usepackage|.\\
\\
\Underline{\|\setcounter{|{\bf volume}\|}{<volume>}|}\\
\Underline{\|\setcounter{|{\bf number}\|}{<number>}|}\\
\Underline{\|\setcounter{|{\bf page}\|}{<first-page>}|}\\
\Underline{\|\|{\bf received}\|{<year>}{<month>}{<day>}|}\\
\Underline{\|\|{\bf accepted}\|{<year>}{<month>}{<day>}|}\\
\quad Define your own macros if necessary.\\\\
\quad If you cannot use the \|txfonts| package, please do not use the following command\\
\Underline{\|\usepackage[varg]{txfonts}|}\\
\Underline{\|\makeatletter|}\\
\Underline{\|\input{ot1txtt.fd}|}\\
\Underline{\|\makeatother|}\\\\
\|\begin{document}|\\[.5em]
\|\title{<title>}|\\[.5em]
\Underline{\|\affiliate{<affiliation-label>}{<affiliation>}|}\\\\
\quad Declare current affiliation with \|\paffilabel| if necessary.\\
\Underline{\|\paffiliate{<affiliation-label>}{<affiliation>}|}\\\\
\Underline{\|\author{1st-author}{affiliation-label}[E-mail]|}\\
\Underline{\|\author{2nd-author}{affiliation-label}|}\\\\
\|\begin{abstract}|\\
\quad\|<abstract>|\\
\|\end{abstract}|\\\\
\|\begin{keyword}|\\
\quad\|<keyword>|\\
\|\end{keyword}|\\\\
\|\maketitle|\\\\
\|\section{|heading-of-1st-section\|}|\\
\dots\dots\dots\dots\dots\\
\quad \|<main text>|\\
\dots\dots\dots\dots\dots\\\\
\quad Put acknowledgments here with the acknowledgment environment if any.\\
\|\begin{acknowledgment}|\\
\|\end{acknowledgment}|\\\\
\|\begin{thebibliography}{99}%9 or 99|\\
\|\bibitem{1}|\\
\|\bibitem{2}|\\
\|\end{thebibliography}|\\\\
\quad Put appendices here following \|\appendix| if any.\\
\|\appendix|\\
\|\section{|heading-of-1st-section\|}|\\\\
\Underline{\|\begin{biography}|}\\
\Underline{\|\profile{<1st-author>}{<biography-of-1st-author>}|}\\
\Underline{\|\profile{<2nd-author>}{<biography-of-2nd-author>}|}\\
\Underline{\|\end{biography}|}\\
\|\end{document}|

%4.1
\subsection{Option Style}

The following six styles are available as optional arguments of the
\|\documentclass|.  If the JIP option is not used, the program will use
the standard Japanese paper style as the default.

\begin{enumerate}
\item\|JIP| For English documents
\item\|draft| For draft versions
\item\|invited| For invited papers
\item\|sigrecommended| For a paper recommended by a SIG
\item\|technote| For technical notes
\item\|preface| For the preface of an issue
\end{enumerate}

Any combination of these options can be used.

If you use \|\documentclass[JIP,draft]{ipsj}|, the ``draft'' option
style will be applied.  If you specify auxiliary style files with the
\|\usepackage|, you must include them in the file package when you send
your final version to the IPSJ\@.

However, style files included in the {\LaTeXe} standard distribution
(e.g., graphicx) may be omitted.  Note that style files may be
incompatible with the style of the Journal Transaction.

%4.2
\subsection{Title, Author Names, etc.}

Describe the title of your paper, author names and affiliations, and
abstract using the commands and environment shown in Section~\ref{body}.
Then, perform \verb+\maketitle+ to automatically put them at the
appropriate position.  In the draft version, the title and abstract are
automatically printed onto separate pages, while author names and
affiliations are not printed in order to make your paper anonymous.

%4.2.1
\subsubsection{Title}

The title specified with \verb+\title+ is made centered.  Even if the
title is too long to fit onto one line, \textit{an automatic line break
is not performed}.  If your title is long, insert \verb+\\+ into the
appropriate positions to break the lines.  A multiple line title is
first flushed left and then centered with respect to the widest line.

The title also appears in the header of odd numbered pages.  If your
title is too long, provide a shortened title for the header to
\verb+\title+ as its optional argument as follows.

%4.2.2
\subsubsection{Author Name and Affiliation}

When indicating the affiliation of each author with a label (first
citation) and starting from the first author, by using \|\affiliate|,
numbered footnotes will be generated that show the affiliations.  When
several authors are affiliated with the same organization, the
affiliation needs to be indicated only once.  For the author's current
affiliation, use \|\paffiliate| and provide the label and affiliated
organization as before.  If the affiliated organization arguments are
entered as current and a line break is inserted using \|\\|, the author
name will be automatically defined by \|\author|.  Immediately after the
author's name, enter the affiliation label and the author's e-mail
address.

Where there are several authors, repeating \|\author| will generate
additional authors in sequence (two authors, three authors, and so
forth).

To add current affiliations or multiple affiliations, delineate the
affiliate label using commas to include additional data.

%4.2.3
\subsubsection{Abstract}

The abstract of your paper should only be used in the \verb+abstract+
environment.

%4.2.4
\subsubsection{Keywords}

The keywords of your paper should be included as the content for the
 \verb+keyword+ environment.

%5
\section{Main part}
%5.1
\subsection{Sectioning}

{\LaTeX} standard commands such as \|\section| and \|\sub-| \|section|
are available for sectioning. The section heading of \|\section|
occupies two lines, while others are put into one line.

%5.2
\subsection{Fixed Baselines}

Each page of the JIP is formatted with the double-column style.  The
printing tradition of double-column requires that a line in the left
column and its neighbor in the right column have the same baseline.  To
meet this requirement, the style files carefully control the progression
of baselines when a vertical space is inserted for section titles and so
on.

%5.3
\subsection{Font Size}

You will see that various size fonts are used in the printed result of
your paper.  Since these fonts are automatically and carefully chosen by
the style files, you are free from the headache of selecting proper
fonts.  In fact, it is strongly recommended not to use
font-size-changing commands such as \verb+\large+ and \verb+\small+ in
the main text because they are quite harmful to retaining fixed
baselines.

%5.4
\subsection{Itemizing}\label{sec:item*}

There is no special format for itemization. You can use the standard
\|enumerate|, \|itemize|, \|description| environment.

%5.5
\subsection{Footnotes}

The command \|\footnote| produces footnotes with reference marks such as
\footnote{An example of footnote 1.} and \footnote{An example of
footnote 2.}.  When there is more than one footnote within a single
page, please note that it is necessary to run \LaTeX\ twice to process
them correctly.  Moreover, it is sometimes preferable to separate a
footnote and its mark into different columns.  This can be achieved
using the \|\footnotemark| and \|\footnotetext| commands. The footnote
numbering produced by \LaTeX\ is continuous throughout the paper; it
does not restart on each new page.

%5.5.1
\subsubsection{Overfulls and Underfulls}

The final result must be free from any overfulls.  It is well known that
almost all overfulls can be avoided with a little effort when writing
sentences.  For example, avoiding long in-text formulas and \|\verb| is
very effective.  However, tricks using the \|flushleft| environment,
\|\\|, or \|\linebreak| are not recommended because they cause quite
unattractive results.

For underfulls, you will conveniently get the following warning message,
\begin{quote}\footnotesize*
\|Underfull| \|\hbox| \|(badness 10000)| \|detected|
\end{quote}
, by inserting \|\\| at the end of a paragraph.  This message is also
output when you use \|\\| just before a list-like environment, just
before an \|\item|, and at the end of the environment. Such underfulls
cause unattractive empty lines and a flood of warnings that will hide
important error messages.

%5.6
\subsection{Formulas}\label{sec:ITEM}
%5.6.1
\subsubsection{In-text Formulas}

In-text formulas may be surrounded by any proper math-open\slash close
pair, i.e. \|$| and \|$|, \|\(| and \|\)|, or \|\begin| and \|\end| for
the \|math| environment. Note that tall materials in in-text formulas,
such as \smash{$\frac{a}{b}$} (\|\frac{a}{b}|), are unattractive and
will disarrange the baseline progression.

%5.6.2
\subsubsection{Displayed Formulas}

Displayed formulas {\em must not be surrounded by the pair
\|$$|}.  Instead, use the \|\[| and \|\]| pair or one of the environments
\|displaymath|, \|equation|, or \|eqnarray|.  These commands\slash
environments indent formulas (not centered) and keep fixed baselines as
follows.
\begin{equation}
\Delta_l = \sum_{i=l+1}^L\delta_{pi}.
\end{equation}

%5.6.3
\subsubsection{Eqnarray environment}

For a sequence of two or more related formulas (equations), use the
\|eqnarray| environment to line them up at equal (or unequal) signs
instead of \|\[| \ \|\]| or the \|equation| environment.

%5.6.4
\subsubsection{Special Fonts}

It is strongly recommended to use only standard {\LaTeX} math
fonts.  Otherwise, you must report that you are using special fonts.

\begin{figure}[tb]%1
\setbox0\vbox{\it
\hbox{\|\begin{figure}[tb]|}
\hbox{\quad \|<|figure-body\|>|}
\hbox{\|\caption{<|caption\|>}|}
\hbox{\|\label{| $\ldots$ \|}|}
\hbox{\|\end{figure}\|}}
\centerline{\fbox{\box0}}
\caption{Single column figure with caption\\
explicitly broken by $\backslash\backslash$}
\label{fig:single}
\end{figure}

\begin{figure}[tb]%2
\begin{minipage}[t]{0.5\columnwidth}
\footnotesize
\setbox0\vbox{
\hbox{\|\begin{minipage}[t]%|}
\hbox{\|  {0.5\columnwidth}|}
\hbox{\|\captionType{table}|}
\hbox{\|\caption{| \ldots \|}|}
\hbox{\|\ecaption{| \ldots \|}|}
\hbox{\|\label{| \ldots \|}|}
\hbox{\|\makebox[\textwidth][c]{%|}
\hbox{\|\begin{tabular}[t]{lcr}|}
\hbox{\|\hline\hline|}
\hbox{\|left&center&right\\\hline|}
\hbox{\|L1&C1&R1\\|}
\hbox{\|L2&C2&R2\\\hline|}
\hbox{\|\end{tabular}}|}
\hbox{\|\end{minipage}|}}
\hbox{}
\centerline{\fbox{\box0}}
\caption{Contents of Table \protect\ref{tab:right}}
\label{fig:left}
\end{minipage}%
\begin{minipage}[t]{0.5\columnwidth}
\CaptionType{table}
\caption{A table built by Fig.\ \protect\ref{fig:left}}
\label{tab:right}
\makebox[\textwidth][c]{\begin{tabular}[t]{lcr}\hline\hline
left&center&right\\\hline
L1&C1&R1\\
L2&C2&R2\\\hline
\end{tabular}}
\end{minipage}
\end{figure}

\begin{figure}[t]
\setbox0\vbox{\it
\hbox{\|\begin{figure}[tb]|}
\hbox{\quad \|<|figure-body\|>|}
\hbox{\|\caption{<|caption\|>}|}
\hbox{\|\label{| $\ldots$ \|}|}
\hbox{\|\end{figure}\|}}
\centerline{\fbox{\box0}}
\caption{Single column figure with caption\\
explicitly broken by $\backslash\backslash$}
\end{figure}

\begin{figure*}[t]
\setbox0\vbox{\large
\hbox{\|\begin{figure}*[t]|}
\hbox{\quad\|<|figure-body\|>|}
\hbox{\|\caption{<|caption\|>}|}
\hbox{\|\label{| $\ldots$ \|}|}
\hbox{\|\end{figure*}|}}
\centerline{\fbox{\hbox to.9\textwidth{\hss\box0\hss}}}
\caption{Double column figure}
\label{fig:double}
\end{figure*}

%5.7
\subsection{Figures}

A figure fit to one column is specified by the form shown in
\figref{fig:single}. Note that you must not specify the \|h| option.

The \|\caption| of a figure should be given below the figure body
together with a \|\label| command.  A long caption will be automatically
broken into two or more lines and centered with respect to the widest
line.  You can assist, however, with the line breaking by adding \|\\|
to obtain a more beautiful result, especially for two-line captions, as
shown in \figref{fig:single}.

If you want to rank two or more figures and/or tables in a \|figure| (or
\|table|) environment in order to save space, enclose each figure\slash
table and its \|\caption| in a \|minipage| environment as shown in
\figref{fig:left} and \tabref{tab:right}. Also, as in a \|figure|
environment, the caption for \tabref{tab:right} is correctly typeset
because the \|minipage| for it has the \|\captionType{table}| command to
specify the type of caption.  The command can of course be used with the
\|figure| argument to give a figure caption.

\Figref{fig:double} shows how to make a double column figure.

You may use any size font, as shown in \figref{fig:double}. Also, you
may include an encapsulated PostScript file (so called EPS file) as the
body of a figure. To include, use
%
\begin{quote}
\|\usepackage{graphicx}|
\end{quote}
%
in the preamble and put the \|\includegraphics| command where you wish
to embed the EPS graphics with its file name (and options if necessary). 

You might have noticed that the first reference to \figref{fig:single}
is bold-faced, while the second and third are typed in roman fonts.
This font switching is a rule of the Journal\slash Transactions and will
be automatically performed if you use \|\figref{<|label\|>}| instead of
\|Fig.~\ref{<|label\|>}|.  Another rule is that ``Figure'' must be used
instead of ``Fig.''\ if the reference is the first word of a sentence,
such as was the first reference to \figref{fig:double} above.
Unfortunately, this switching is too hard to do automatically, so you
must use \|\figref{<|label\|>}| in such cases.

%5.8
\subsection{Tables}

A table with many rules is not very beautiful. \tabref{tab:example}
shows an example of a table with standard style rules. Note that the
uppermost rule is doubled, and no rules are drawn on the left and right
edges. The caption should be put above the table. The default font size
for tables is \|\footnotesize|. Any reference to a table should be made
using \|\tabref{<|label\|>}|.

\begin{table}[tb]
\caption{Sections and sub-sections in which list-like environments are used (example of table)}
\label{tab:example}
\hbox to\hsize{\hfil
\begin{tabular}{l|lll}\hline\hline
&enumerate&itemize&description\\\hline
type-1&	2 & 3 & 4.5 \\
type-2&	---& 4.11 & 4.7 \\
type-3&	2 & --- & 4.5\\
type-4&	--- & 4.8 & 4.3 \\\hline
\multicolumn{4}{l}{type-1\,: {\tt enumerate}, etc.\quad
	type-2\,: {\tt enumerate*}, etc.}\\
\multicolumn{4}{l}{type-3\,: {\tt Enumerate}, etc.\quad
	type-4\,: {\tt ENUMERATE}, etc.}\\
\end{tabular}\hfil}
\end{table}

%5.9
\subsection{Citations, Reference, Acknowledgements}
%5.9.1
\subsubsection{Citations}

The command \|\cite| is used to add citations in the text.  Cited labels
are sorted automatically and separated by using square brackets \|[ ]|.
Thus,
\begin{quote}
\|The paper \cite{companion,latex} is|\\
\|an overview of \LaTeX|.
\end{quote}
will produce
\begin{quote}
The paper \cite{companion,latex} is an overview of \LaTeX.
\end{quote}

%5.9.2
\subsubsection{List of References}


References should be arranged in alphabetical or cited order.
It is recommended to use BiB{\TeX} and style files 
\|ipsjsort-e.bst|
(alphabetical order) or \|ipsjunsort-e.bst| (cited order) to make
references fit to the traditional style.
Remember that you must include \|.bbl| file in the file package, instead of
\|.bib|.
If you cannot use BiB{\TeX} and have to make references manually using the
bibliography environment, observe the references of this guide carefully
and follow its style.





%5.9.3
\subsubsection{Acknowledgments and Appendices}

If you want to acknowledge people, put your acknowledgments just before
the references and enclose them in the \|acknowledgment|
environment. Acknowledgments will not be printed in drafts.

Appendices, if there are any, should be put just after the references
and \|\appendix| command.  Sectioning commands produces headings like
{\bf \ref{A1}}, {\bf \ref{A2}}, and so on in the appendices.

%5.10
\subsection{Biography}

Biographies of authors are positioned at the end of the document, just
before \|\end{document}|, as follows.
%
\begin{quote}
\|\begin{biography}|\\
\|\profile{<|1st-author's-name\|>}|\\
\mbox{}\quad\|{<|biography-of-1st-author\|>}|\\
\|\author{<|2nd-author's-name\|>}|\\
\mbox{}\quad\|{<|biography-of-2nd-author\|>}|\\
\mbox{}\quad $\ldots\ldots\ldots$ \\
\|\end{biography}|
\end{quote}

%6
\section{Check List of ``Dos and Don'ts''}
%6.1
\subsection{The basics of writing}

\begin{itemize}
\item[$\Box$] Describe a paper so that readers understand the novelty,
	      availability, and reliability of the research.
\item[$\Box$] Try to make a paper easy to read (discontinuity in the
	      story and obscure backgrounds or themes are a burden to
	      readers). 
\item[$\Box$] Revisit the paper if the problem to be solved is not
	      generalized (entirely focused on a problem at XX
	      University, etc.) or if the paper reports deliverables
	      only and does not describe the problem itself. 
\item[$\Box$] Rethink the paper if its conclusion is not clearly
	      described, it does not adequately point out its
	      applicability, limits, and controversial points, or its
	      conclusion does not follow the contents. 
\item[$\Box$] Expressions that are inappropriate for scientific papers
	      and that are hard to understand should be reconsidered. 
\item[$\Box$] Second thought is necessary if sentences are in colloquial
	      style. 
\item[$\Box$] Check the structure of chapters and sections and the
	      organization of the paper. 
\item[$\Box$] Do not make the paper so that grasping the meaning is
	      difficult without guessing from the context. 
\item[$\Box$] Confirm if the explanation of the hypotheses is enough and
	      does not contain any gaps in meaning. 
\item[$\Box$] The authors should not submit a manuscript that includes
	      redundant and/or too brief descriptions.
\item[$\Box$] The authors should eliminate undefined terminologies.
\end{itemize}

%6.2
\subsection{Show novelty and usefulness clearly}

\begin{itemize}
\item[$\Box$] The authors should not submit a manuscript that does not
	      clarify the motivation and the goal of their study and the
	      relationship to other existing studies.
\item[$\Box$] The authors should not submit a manuscript that does not
	      clarify what technologies are well/publicly known and what
	      idea they are newly/originally proposing. 
\item[$\Box$] The authors should provide sufficient references in their
	      manuscript to back up the originality of their study. 
\item[$\Box$] The authors should not submit a manuscript in which the
	      readers cannot understand their proposal (or cannot find
	      any originality in it) because it consists entirely of
	      abstractive and/or conceptual descriptions. 
\item[$\Box$] The authors should not submit a manuscript that lacks
	      discussions on the effectiveness of their proposal. 
\end{itemize}

%6.3
\subsection{Concrete attention to writing}

\begin{itemize}
\item[$\Box$] The authors should not submit a manuscript whose Japanese
	      title does not match its content correctly.
\item[$\Box$] The authors should not submit a manuscript whose English
	      title does not match its content correctly or that
	      contains incorrect English usage.
\item[$\Box$] The paper should be revised when its abstract does not
	      show its purpose or is written in inadequate English.
\item[$\Box$] The paper should be revised when symbols and abbreviations
	      are not popular, wordings are not adequate, or the
	      explanations of its pictures and tables are not adequate.
\item[$\Box$] The paper should be revised when special wordings, which
	      are popular only in an individual or local group or a
	      small company, are used without any explanations.
\item[$\Box$] The paper should be revised when its pictures or tables
	      are not semantically clear or they contain mistakes.
\item[$\Box$] The paper should be revised when its pictures or tables
	      are not visually clear.
\item[$\Box$] The paper should be revised when the size or the scale of
	      its pictures or tables are not adequate.
\end{itemize}

%6.4
\subsection{Regarding references}

\begin{itemize}
\item[$\Box$] The number of references should be more than 10 (Some
	      opinions say more than 20 or 30 in some research areas.
\item[$\Box$] A sufficient number of references are required to show the
	      paper's novelty.
\item[$\Box$] The paper should be revised when it has an insufficient
	      number of references.
\item[$\Box$] Referring to appropriate papers written by Japanese
	      authors contributes to the further progression of the
	      Japanese research community.
\item[$\Box$] Do not include self-citations excessively.
\end{itemize}

%6.5
\subsection{Double submission}

\begin{itemize}
\item[$\Box$] Double submission of the original paper is
	      prohibited. However, it is permissible to submit a paper
	      accepted at an international conference and free from
	      copyright issues.
\item[$\Box$] Do not use the same figures or charts already included in
	      other original papers, except those that have proper
	      citations.
\item[$\Box$] Be careful not to have overlap between the paper and other
	      published articles.
\end{itemize}

%6.6
\subsection{Check by other researchers}

\begin{itemize}
\item[$\Box$] Proofreading by experienced persons with many accepted
	      papers is strongly recommended.
\item[$\Box$] Take care to avoid leaps of logic from the viewpoint of
	      the readers.
\end{itemize}

%6.7
\subsection{Miscellaneous}

\begin{itemize}
\item[$\Box$] After the first review round, do not modify the paper
	      except for the stated conditions for acceptance without
	      the reviewers' approval.
\item[$\Box$] Since the IPSJ uses a double-blind review system, in which
	      both author(s) and reviewers remain anonymous, the authors
	      cannot select reviewers.
\item[$\Box$] Fill the self-check sheet carefully before submitting the
	      paper.
\end{itemize}

%7
\section{Concluding Remarks}

We dare not dream that the style files are perfect but rather wish to
improve them with your cooperation and hope that you will let us know of
any complaints, comments, suggestions by e-mail to: 
\begin{quote}
\|editt@ipsj.or.jp|.
\end{quote}




\begin{acknowledgment}
We wrote this article based on the guideline for A4 landscape layout.
We are grateful to Prof.\ Hiroshi Nakashima from Kyoto University,
for his valuable comments on making a class-file, 
and his consent to usage of BiB{\TeX} files.
We are also very thankful to the editorial committee for their
contributions in writing the guideline for the A4 landscape layout.
\end{acknowledgment}







\begin{thebibliography}{99}
\bibitem{companion}%1
Goossens, M., Mittelbach, F., and Samarin, A.:
{\it The LaTeX Companion},
Addison Wesley, Reading, 
Massachusetts (1993).

\bibitem{latex}%2
Lamport, L.: 
{\it A Document Preparation System {\LaTeX} User's Guide \&
Reference Manual}, 
Addison Wesley, Reading, Massachusetts (1986).

\bibitem{article1}%3
Itoh, S. and Goto, N.: 
An Adaptive Noiseless Coding for Sources with Big
Alphabet Size, 
{\it Trans.\ IEICE},  
Vol. E74, No. 9, pp. 2495--2503 (1991).

\bibitem{article2}%4
Abrahamson, K., Dadoun, N., Kirkpatrick, D.G., and Przytycka, T.: 
A Simple Parallel Tree Contraction Algorithm, 
{\it J.\ Algorithms},  
Vol. 10, No. 2,
pp. 287--302 (1989).

\bibitem{article3}%5
Yamakami, T.: Exploratory Session Analysis in the Mobile Clickstream, 
{\it IPSJ Digital Courier},  
Vol. 3, pp. 14--20 (online), \\
\doi{10.2197/ipsjdc.3.14} (2007).

\bibitem{book1}%6
Foley, J.D. et al.: 
{\it Computer Graphics --- Principles and Practice},
System Programming Series, Addison-Wesley, 
Reading, Massachusetts, 2nd edition (1990).

\bibitem{book2}%7
Chang, C.L. and Lee, R.C.T.: 
{\it Symbolic Logic and Mechanical Theorem Proving}, 
Academic Press, New York (1973).

\bibitem{booklet1}%8
{Institute for New Generation Computer Technology}: 
Overview of the Fifth Generation Computer Project, 
distributed in {FGCS'92} (1992).
(in Japanese).

\bibitem{inbook1}%9
Knuth, D.E.: 
{\it Fundamental Algorithms}, 
Art of Computer Programming,
Vol. 1, chapter 2, 
pp. 371--381, 
Addison-Wesley, 2nd edition (1973).

\bibitem{incollection1}%10
Schwartz, A.J.: 
Subdividing B{\'e}zier Curves and Surfaces, 
{\it Geometric Modeling: Algorithms and New Trends\/} 
(Farin, G.E., ed.), 
SIAM, Philadelphia,
pp. 55--66 (1987).

\bibitem{inproceedings1}%11
Baraff, D.:
Curved Surfaces and Coherence for Non-penetrating Rigid Body
Simulation, 
{\it SIGGRAPH '90 Proceedings\/} (Beach, R.J., ed.), 
Dallas,
Texas, ACM, Addison-Wesley, 
pp. 19--28 (1990).

\bibitem{inproceedings2}%12
Nakashima, H. et al.: 
OhHelp: A Scalable Domain-Decomposing Dynamic Load
Balancing for Particle-in-Cell Simulations, 
{\it Proc.\ Intl.\ Conf. Supercomputing}, 
pp. 90--99 (online),\\
\doi{http://doi.acm.org/10.1145/1542275.1542293} (2009).

\bibitem{manual1}%13
Adobe Systems Inc.: 
{\it PostScript Language Reference Manual}, 
Reading,
Massachusetts (1985).

\bibitem{mastersthesis1}%14
Ohno, K.: 
Efficient Message Communication of Concurrent Logic Programming
Language KL1 Based on Static Analysis, 
Master's thesis, 
Dept.\ Information Science, Kyoto University (1995).

\bibitem{misc1}%15
Saito, Y. and Nakashima, H.: 
{\tt ipsjpapers.sty} (1995).
(Style file for Trans. IPSJ distributed to authors.).

\bibitem{phdthesis1}%16
Weihl, W.: 
Specification and Implementation of Atomic Data Types, 
PhD Thesis,
MIT, Boston (1984).

\bibitem{proceedings1}%17
Institute for New Generation Computer Technology: 
{\it Proc.\ Intl.\ Conf.\ on Fifth Generation Computer Systems}, 
Vol. 1 (1992).

\bibitem{WarD:WAM-1}%18
Warren, D.H.D.: An Abstract {Prolog} Instruction Set, 
Technical Report 309,
Artificial Intelligence Center, 
SRI International (1983).

\bibitem{unpublished}%19
Editorial Board of Trans.\ IPSJ:
How to Typeset Your Papers in {\LaTeX}
(Version 1) (1995).
(distributed to authors).

\bibitem{webpage1}%20
Kay, A.: Welcome to Squeakland, Squeakland (online),\\
\urle{http://www.squeakland.org/community/biography/\\
alanbio.html}
\refdatee{2007-4-5}.

\bibitem{webpage2}%21
Nakashima, H.: 
A {WEB} Page, Kyoto University (online),\\
\urle{http://www.para.media.kyoto-u.ac.jp/\~{}nakashima/\\
a.web.page.of.long.url/}
\refdatee{2010-10-30}.

\bibitem{webpage3}%22
Nakashima, H.: 
Another {WEB} Page, 
Kyoto University (online),\\
\urle{http://www.para.media.kyoto-u.ac.jp/\~{}nakashima/\\
a.web.page.of.much.longer.url/} 
\refdatee{2010-10-30}.

\end{thebibliography}

\appendix

%8
\section{How to Write an Appendix}
\label{A1}

To add an appendix, write the command \|\appendix| immediately following
the reference list.  Within the appendix, the \|\section| command creates
numbered headings such as \ref{A1} and \ref{A2}.

%8.1
\subsection{Example of a Heading}
The command \|\subsection| in the appendix gives this kind of heading.

%9
\section{Commands for Transactions}
\label{A2}

Each transaction has its own subtitle, abbreviation code, and serial
number. This information is given by using the following commands for
the \|\documentclass| option in the final version.

\begin{itemize}
\item \|PRO| (Trans.\ Programming)
\item \|TOM| (Trans.\ Mathematical Modeling and Its Applications)
\item \|TOD| (Trans.\ Database)
\item \|ACS| (Trans.\ Advanced Computing Systems)
\item \|CDS| (Trans.\ Consumer Device \& System)
\item \|TBIO| (Bioinformatics)
\item \|SLDM| (System LSI Design Methodology)
\item \|CVA| (Computer Vision and Applications)
\end{itemize}

Moreover, for papers in English, the command English can be added. For
example, writing \|\documentclass[PRO,| \|english]{ipsj}| will create an
English document.

Note that the research group has a ``month of publication'' number that
does not correspond to the ``issue month number'' of the
transaction. You may be notified by the IPSJ or the Editorial Board of
the \|<month>| in order to set the month of publication counter as
follows.

\begin{quote}
\|\setcounter{month}{month of publication}|
\end{quote}

In addition, commands are provided for executing unique functions for
several transactions, as shown in the following sections.

%10
\section{Unique Commands for Each Part}

Since each of the parts has its own detailed specifications, the same
command may produce different results in two different parts. 

In some cases, the \|<Revised date>| and \|<Second revised| \|date>| are
inputted. These can be added as a preamble by using 

\begin{quote}
\|\rereceived{<year>}{<month>}{<day>}|\\
\|\rerereceived {<year>}{<month>}{<day>}|
\end{quote}

%10.1
\subsection{Unique Functions for Programming (PRO)}

Issues of Transactions on Programming (PRO) includes not only regular
papers but also abstracts from research presentations delivered in the
research groups of SIGPRO\@. The file for an abstract consists of
material from the \|\documentclass| to the \maketitle of the format
shown in Section~\ref{body}. That is, the file does not have a main
text. Note that the reception and acceptance dates are not required, but
the date of presentation has to be given:
\begin{quote}
\|\Presents{<year>}{<month>}{<day>}|
\end{quote}

%10.2
\subsection{Unique Functions for Database (TOD)}

The name of the editor in charge for the paper included in The
Transactions on Database (TOD) is specified by
\begin{quote}
\|\edInCharge{<name-of-editor>}|
\end{quote}

Also, following a change in style, the command is entered at the end of
the paper, directly before \|\end{document}|.

%10.3
\subsection{Unique Functions for Consumer Devices \& Systems (CDS)}

In the ``Transactions on Consumer Devices \& Systems,'' since the
headings differ depending on the type of document, the type of heading
is to be changed with the option.

The types are:
\begin{itemize}
\item \|systems  | Paper on Consumer Systems
\item \|services | Paper on Consumer Services
\item \|devices  | Paper on Consumer Devices
\item \|research | Research Paper
\end{itemize}
For English papers, you simply need to add English.

%10.4
\subsection{Unique Functions for Bioinformatics (TBIO)}

Since papers in Transactions on Bioinformatics (TBIO) are in English,
specifying the TBIO option will cause the program to assume that the
English option has been specified. This effectively means that the
English option can be omitted.

The following three categories define the different types of papers.
\begin{itemize}
\item \|No specification | Original Paper (Default)
\item \|Data   | Database/Software Paper
\item \|Survey | Survey Paper
\end{itemize}


Therefore, \|\documentclass[TBIO]{ipsj}| will be an original paper, and
\|\documentclass[TBIO,Survey]{ipsj}| will be a survey paper. 

Moreover, as with TOD, the name of the editor in charge of the paper is
specified by using \|\Editor|, but in this case, the text is introduced with ``Communicated by.'' Therefore, the name of the editor is positioned directly before \|\end{document}|, as with TOD.

%10.5
\subsection{Unique Functions for Computer Vision and Applications (CVA)}

The Transactions of Computer Vision and Applications is also an English
language journal, allowing the English option to be omitted.

There are three classes of documents:
\begin{itemize}
\item \|No specification | Regular Paper (Default)
\item \|Research | Research Paper
\item \|system   | Systems Paper
\end{itemize}


As with TBIO, the name of the editor in charge is inserted, and the
inserted text is introduced with ``Communicated by.''

%10.6
\subsection{Unique Functions for System LSI Design Methodology (SLDM)}

The Transactions of System LSI Design and Methodology (SLDM) is also an
English language journal, allowing the English option to be omitted.

There are two classes of documents:
\begin{itemize}
\item \|No specification| Regular Paper (Default)
\item \|Short| Short Paper
\end{itemize}

SDLM also enters the name of the editor in charge, but automatic
insertion is treated differently depending on the paper.

Normally, text is inserted using ``Recommended by Associate Editor:,''
but it is only when the ``invited'' option is included that the
insertion text becomes ``Invited by Editor-in-Chief:.''



%% �ʹ�̵�뤵��ޤ�

\begin{biography}
\profile{Joho Taro}{was born in 1970. He received his M.S.\ degree from
 Johoshori University in 1994 and has been engaged in the Information
 Processing Society of Japan since 1994. His research interest is online
 publishing systems. He is a member of the IEEE and ACM\@.}
%
\profile{Shori Hanako}{was born in 1960. She received her M.E.\ and
 Ph.D.\ from Johoshori University in 1984 and 1987, respectively. She
 became an associate professor at Gakkai University in 1992 and a
 professor at Johoshori University in 1997. Her current research
 interest is online publishing systems. She received the Kiyasu Kinen
 award in 2010. She is a Board Member of the IPSJ and a member of the
 IEICE, IEEE-CS, and ACM\@.} 
%
\profile{Gakkai Jiro}{was born in 1970. He received his M.S.\ degree
 from Johoshori University in 1994 and has been engaged in the
 Information Processing Society of Japan since 1994. His research
 interest is online publishing systems. He is a member of the IEEE and
 ACM\@.}
%
\end{biography}
\end{document}
