%%「論文」,「レター」,「レター(C分冊)」,「技術研究報告」などのテンプレート
%% 1. 「論文」
%% v1.6 [2009/11/03]
\documentclass{ieicej}
%\documentclass[invited]{ieicej} % 招待論文
%\documentclass[comment]{ieicej} % 解説論文
\usepackage{graphicx}
%\usepackage{latexsym}
%\usepackage[fleqn]{amsmath}
%\usepackage[psamsfonts]{amssymb}

\setcounter{page}{1}

\field{}
\jtitle{}
\etitle{}
\authorlist{%
 \authorentry{}{}{}% <= 記述しないとエラーになります
 %\authorentry{和文著者名}{英文著者名}{所属ラベル}
 %\authorentry[メールアドレス]{和文著者名}{英文著者名}{所属ラベル}
 %\authorentry{和文著者名}{英文著者名}{所属ラベル}[現在の所属ラベル]
}
\affiliate[]{}{}
%\affiliate[所属ラベル]{和文所属}{英文所属}
%\paffiliate[]{}
%\paffiliate[現在の所属ラベル]{和文所属}

\begin{document}
\begin{abstract}
%和文あらまし 500字以内
\end{abstract}
\begin{keyword}
%和文キーワード 4〜5語
\end{keyword}
\begin{eabstract}
%英文アブストラクト 100 words
\end{eabstract}
\begin{ekeyword}
%英文キーワード
\end{ekeyword}
\maketitle

\section{まえがき}


\ack %% 謝辞

%\bibliographystyle{sieicej}
%\bibliography{myrefs}
\begin{thebibliography}{99}% 文献数が10未満の時 {9}
\bibitem{}
\end{thebibliography}

\appendix
\section{}

\begin{biography}
\profile{}{}{}
%\profile{会員種別}{名前}{紹介文}% 顔写真あり
%\profile*{会員種別}{名前}{紹介文}% 顔写真なし
\end{biography}

\end{document}



%% 2. 「レター」
\documentclass[letter]{ieicej}
\usepackage{graphicx}
%\usepackage{latexsym}
%\usepackage[fleqn]{amsmath}
%\usepackage[psamsfonts]{amssymb}

\setcounter{page}{1}

\typeofletter{研究速報}
%\typeofletter{紙上討論}
%\typeofletter{問題提起}
%\typeofletter{ショートノート}
\field{}
\jtitle{}
\etitle{}
\authorlist{%
 \authorentry{}{}{}{}% <= 記述しないとエラーになります
 %\authorentry{和文著者名}{英文著者名}{会員種別}{所属ラベル}
 %\authorentry{和文著者名}{英文著者名}{会員種別}{所属ラベル}[現在の所属ラベル]
}
\affiliate[]{}{}
%\affiliate[所属ラベル]{和文所属}{英文所属}
%\paffiliate[]{}
%\paffiliate[現在の所属ラベル]{和文所属}

\begin{document}
\maketitle
\begin{abstract}
%和文あらまし 120字以内
\end{abstract}
\begin{keyword}
%和文キーワード 4〜5語
\end{keyword}
\begin{eabstract}
%英文アブストラクト 50 words
\end{eabstract}
\begin{ekeyword}
%英文キーワード
\end{ekeyword}

\section{まえがき}


\ack %% 謝辞

%\bibliographystyle{sieicej}
%\bibliography{myrefs}
\begin{thebibliography}{99}% 文献数が10未満の時 {9}
\bibitem{}
\end{thebibliography}

\appendix
\section{}

\end{document}


%% 3. 「レター(C分冊)」
\documentclass[electronicsletter]{ieicej}
\usepackage{graphicx}
%\usepackage{latexsym}
%\usepackage[fleqn]{amsmath}
%\usepackage[psamsfonts]{amssymb}

\setcounter{page}{1}

\field{}
\jtitle{}
\etitle{}
\authorlist{%
 \authorentry{}{}{}{}% <= 記述しないとエラーになります
 %\authorentry{和文著者名}{英文著者名}{会員種別}{所属ラベル}
 %\authorentry{和文著者名}{英文著者名}{会員種別}{所属ラベル}[現在の所属ラベル]
}
\affiliate[]{}{}
%\affiliate[所属ラベル]{和文所属}{英文所属}
%\paffiliate[]{}
%\paffiliate[現在の所属ラベル]{和文所属}

\begin{document}
\begin{abstract}
%和文あらまし 120字以内
\end{abstract}
\begin{keyword}
%和文キーワード 4〜5語
\end{keyword}
\begin{eabstract}
%英文アブストラクト 50 words
\end{eabstract}
\begin{ekeyword}
%英文キーワード
\end{ekeyword}
\maketitle

\section{まえがき}


\ack %% 謝辞

%\bibliographystyle{sieicej}
%\bibliography{myrefs}
\begin{thebibliography}{99}% 文献数が 10 未満の時 {9}
\bibitem{}
\end{thebibliography}

\appendix
\section{}

\end{document}



%% 4. 「技術研究報告」
\documentclass[technicalreport]{ieicej}
\usepackage{graphicx}
%\usepackage{latexsym}
%\usepackage[fleqn]{amsmath}
%\usepackage[psamsfonts]{amssymb}

\jtitle{}
\jsubtitle{}
\etitle{}
\esubtitle{}
\authorlist{%
 \authorentry[]{}{}{}% <= 記述しないとエラーになります
% \authorentry[メールアドレス]{和文著者名}{英文著者名}{所属ラベル}
}
\affiliate[]{}{}
%\affiliate[所属ラベル]{和文勤務先\\ 連絡先住所}{英文勤務先\\ 英文連絡先住所}

\begin{document}
\begin{jabstract}
%和文あらまし
\end{jabstract}
\begin{jkeyword}
%和文キーワード
\end{jkeyword}
\begin{eabstract}
%英文アブストラクト
\end{eabstract}
\begin{ekeyword}
%英文キーワード
\end{ekeyword}
\maketitle

\section{はじめに}


%\bibliographystyle{sieicej}
%\bibliography{myrefs}
\begin{thebibliography}{99}% 文献数が10未満の時 {9}
\bibitem{}
\end{thebibliography}

\end{document}
