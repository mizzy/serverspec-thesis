\section{提案手法の評価}

提案手法を実装したserverspecを採用している企業がいくつか見受けられる.また,同種ツールとしてとりあげたTest Kitchenやrspec-systemには,ツール標準のテスト機構をserverspecで置き換えるbusser-serverspecやrspec-system-serverspecが存在する.他にも,Vagrantと連携してVMのテストを行うvagrant-serverspecというツールが存在する.このように,serverspecは単体利用だけではなく,同種ツールの一機能として取り込まれたり,他種ツールと連携する形でも利用が広がっている.

serverspecがなぜ同種のツールと比べて使われているのかを考察する.まずひとつは特定の構成管理ツールに依存していないことが挙げられる.そのためどの構成管理ツールを利用していてもserverspecは利用できるため,特定の川迫罹患利ツール依存のツールと比べて利用の間口が広い.また,構成管理ツールを利用していない場合ですら,serverspecを利用することができる.

特定の構成管理ツールに依存していないということは,テスト対象のサーバに特定の構成管理ツールを入れる必要がないということでもある.テスト対象サーバでsshdが動いてさえいれば,Rubyすら入れる必要がない.そのため特定の構成管理ツール依存のテストツールと比較して利用の障壁が低い.serverspecはテスト対象のサーバにSSHログインできさえすれば良く,Rubyすら入っている必要がない.

広く使われている理由の二つ目は記法の抽象度の高さである.抽象度が高いためテストを容易に書くことができ,テストコードの見通しが良く,メンテナンス性が高い.

構成管理ツール依存を排除し,汎用的なのに加え,単機能でもある.ゆえに他ツールとも組み合わせやすく,従来からあるテストツールに取り込まれる形での利用も広がっている.

弊社ではserverspecを利用することによって云々.

とは言え,利用の広がりについては,きちんとした調査の結果ではなく,定量的な評価は今後の課題である.
