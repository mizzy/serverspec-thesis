\section{サーバの構成管理とテスト手法}

本論文で提案するストフレームワークは,サーバ構成を記述したコードのテストをいかに効率よく行うか,という観点から出発している.そこで代表的な構成管理ツールを例に,ツールと言語の特徴,テスト手法ならびに従来手法の問題点について言及する.

\subsection{CFEngineからChefへ}

安価なUNIXライクOSを搭載したサーバが普及し,TCP/IPにより異なるOSを搭載したサーバ同士がネットワーク接続可能になったことにより,システムが大規模化・複雑化したことで,シェルスクリプトに代わり,サーバの設定を宣言的なコードで扱う構成管理手法が提案され,その実装としてCFEngine\cite{cfengine}が登場した.

2005年にはCFEngineに影響を受け,発展させたPuppet\cite{puppet}が登場,2009年にはPuppetから影響を受けたChef\cite{chef}が登場している.CFEngine,Puppetは独自のDSLでサーバの構成を記述するが,ChefはRubyによる内部DSLで記述する.その特性ゆえ,Chefはシステム管理者よりも開発者に強く訴求し,Amazon EC2の様な開発者自身がサーバ構築・運用を行える環境の普及とともに,広まりを見せている.

\subsection{ChefからTest-Driven Infrastructureへ}

CFEngineやPuppetでは,記述できることが独自DSLの範囲内に収まるため,コードは比較的簡易なものとなるが,ChefはRubyでできることはどんなことでも記述できるため,システムが複雑になるにともない,それを記述するコードも複雑になる.そこでサーバ構成を記述したコードに対し,アジャイル開発におけるテスト駆動開発のプロセスを適用する事で,効率よくコードを記述することができる,といった考えが生まれる.Test-Driven Infrastuctureという概念はChefコミュニティ周辺で発生したものであるが,それは偶然ではなく,このようなChefが持つ言語特性ゆえである.

