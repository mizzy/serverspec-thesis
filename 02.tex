\section{サーバの構成管理とテスト手法}

安価なUNIXライクOSを搭載したサーバが普及し,TCP/IPにより異なるOSを搭載したサーバ同士がネットワーク接続可能になったことにより,システムが大規模化・複雑化したことで,シェルスクリプトに代わり,サーバの設定を抽象的なコードで扱う構成管理手法が登場した.\cite{cfengine paper1}

