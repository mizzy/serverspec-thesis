\section{まとめ}

本論文では,Test-Driven Infrastructureを支援するための従来テストフレームワーク手法の問題点について指摘し,解決するために必要な要件として,構成管理ツール独立性とOS・ディストリビューション汎用性を提案した.また,これらの要件を満たすために,汎用コマンド実行フレームワークと制御テストフレームワークに分離して定義する手法についても提案した.これらふたつの実装例としてSpecInfraとserverspecを紹介した.

serverspecの登場によってTest-Driven Infrastuctureというプロセスが国内でも徐々に認知され,この分野の今後の発展が期待される.またSpecInfraを基盤とした,serverspecよりも優れたテストフレームワーク実装の登場や,確認作業以外の運用業務に必要なコマンド群を体系化することによるテスト以外へのSpecInfraの応用,例えば構成管理ツールへの応用なども今後期待される.

今後の課題として,提案手法による運用効率の向上の定量的評価ができていないので,評価手法について検討するとともに,手法に関する助言を募りたい.

