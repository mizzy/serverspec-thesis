\section{まとめ}

本論文では,Test-Driven Infrastructureを支援するための従来テストフレームワーク手法の問題点について指摘し,解決するために必要な要件として,構成管理ツール独立性とOS・ディストリビューション汎用性を提案した.また,これらの要件を満たすために,汎用コマンド実行フレームワークと制御テストフレームワークに分離して定義する手法についても提案した.これらふたつの実装例としてSpecInfraとserverspecを紹介した.

serverspecの登場によってTest-Driven Infrastuctureというプロセスが国内でも徐々に認知され,この分野の今後の発展が期待される.またSpecInfraを基盤としたserverspecよりも優れた実装の登場や,多種ツールへのSpecInfraの応用(例えば構成管理ツールなど)も今後期待される.

今後の課題として,提案手法による運用効率の向上の定量的評価ができていないので,評価手法について検討するとともに,手法に関する助言を募りたい.


