\documentclass[submit,techreq]{ipsj}


\usepackage[dvipdfmx]{graphicx}
\usepackage{latexsym}

\def\Underline{\setbox0\hbox\bgroup\let\\\endUnderline}
\def\endUnderline{\vphantom{y}\egroup\smash{\underline{\box0}}\\}
\def\|{\verb|}

\begin{document}


\title{serverspec: 宣言的記述でサーバの状態をテスト可能な\\
汎用性の高いテストフレームワーク}

\etitle{serverspec: A Versatile Test Framework for Testing States of Servers by Delarative Description}

\paffiliate{PAPERBOY}{株式会社paperboy\&co.\\
paperboy\&co., Inc.
}

\paffiliate{TU}{帝京大学 理工学部 情報科学科 通信教育課程\\
Department of Information Science Correspondence Course, Faculty of SCIENCE and ENGINEERING, Teikyo University
}

\paffiliate{KU}{京都大学 情報科学研究科,京都市\\
Graduates School of Infromatics, Kyoto University
}

\author{宮下 剛輔}{Gosuke Miyashita}{PAPERBOY,TU}[gosukenator@gmail.com]
\author{栗林 健太郎}{Kentaro Kuribayashi}{PAPERBOY}
\author{松本 亮介}{Ryosuke Matsumoto}{KU}

\begin{abstract}
システムの大規模・複雑化に伴い,サーバの構築・運用を効率化するために,サーバの状態をコードで記述する手法が数多く提供されている.それらの手法を効率良く扱うプロセスとして,テスト駆動開発の手法をサーバ構築に応用したTest-Driven Infrastructureが提案されている.このプロセスを支援するテストフレームワークもいくつか登場しているが,あるものは特定の構成管理ツールに依存,またあるものはOS毎の違いを自ら吸収しなければならないなど,汎用性に難がある.そこで,本論文では,特定の構成管理ツールやOSに依存することなく,サーバの状態を汎用的かつ可読性の高いコードでテスト可能なテストフレームワークを提案する.提案手法では,汎用性を高めるために,これまでのOSや構成管理ツール固有の振る舞いを整理して一般化し,運用業務で発生するコマンド群,特に確認作業に関するコマンド群を体系化・抽象化した汎用コマンド実行フレームワークを定義する.続いて,テストコード記述の抽象度を高め可読性を上げるために宣言的かつ自然言語に近い記法で汎用コマンド実行フレームワークを操作できる制御テストフレームワークを定義する.これにより,管理者がOSや構成管理ツールの違いを気にすることなくサーバの状態を容易にテストできるようになり,サーバの運用・管理コストを低減できる.また,フレームワークを用途別に分離して定義することにより,制御テストフレームワークを独自の記述に変更する事も容易である.提案するテストフレームワークをserverspecと名付けた.
\end{abstract}

\begin{eabstract}
As the increase of large and complex systems, many ways to describe states of servers as code are proposed.As the effective process to handle these ways, Test-Driven Infrastructure is proposed.It is the application of Test-Driven Development to the server configuration.Several test frameworks that support this process are appearing, but they have difficulty in vesatility because some frameworks depend on specific configuration management tools and others need the consideration of differences between OSes to write test code.In this paper, we propose a test framework that doesn't depend on specific configuration management tools or OSes for testing states of servers by versatile and readable code.In our proposal, we arrange and generalize the behavior of OSes and configuration management tools and define the versatile command execution framework for increasing the versatility. Next, we define the test framework that control the command execution framework by declarative description to increase the level of abstraction and readability.By this test framework, system administrators can test states of servers easily without considering the diffrences between OSes or configuration management tools, and can decrease the costs of server operations.Also by defining the frameworks separated into purposes, we can change the control test framework to another one that have original notation easily.We named the proposed test framework serverspec.
\end{eabstract}


\maketitle

\section{はじめに}

サーバの構築・運用を効率化するために,サーバの構成管理方針を宣言的なコードで記述するという手法が提案され,その実装として構成管理ツールCFEngine\cite{cfengine}が1993年に登場している.また,昨今のシステムの大規模化・複雑化の流れや,IaaSのようなサーバの構築・破棄が何度でも手軽にできるような環境の登場を背景に,CFEngineを更に発展させた実装として,2006年にPuppet\cite{puppet}が登場,その後も様々な実装が登場している\cite{chef}\cite{saltstack}\cite{ansible}.

2006年のPuppet登場以降,サーバインフラをコードで記述するという``Infrastructure as Code''という概念が台頭し,サーバインフラのコード化により,``Agile infrastructure and operations''\cite{agile infrastructure}という流れが生まれている.これはアジャイル開発の手法をサーバインフラの構築や運用に適用しよう,という潮流である.

この流れの中で,``Agile infrastructures and operations''を効率よく実践するためのプロセスとして,テスト駆動開発の手法をサーバインフラの構築・運用に応用した``Test-Driven Infrastructure''\cite{test driven infrastructure with chef}というプロセスが提案されている.

このプロセスを支援するテストフレームワークがいくつも登場している\cite{chefspec}\cite{rspec-puppet}\cite{cucumber-chef}\cite{minitest-chef-handler}\cite{test kitchen}\cite{rspec-system}.

これらのうち,ChefSpec,rspec-puppetは,構成管理ツール固有の言語で書かれたコードの内容をテストするのみで,実際にコードをサーバに適用した結果はテストしない.Cucumber-chef,minitest-chef-handlerはChefという特定の構成管理ツールに完全に依存している.Test Kitchenやrspec-systemは,テスト用サーバの構築,テスト用サーバへの構成管理ツールの適用,テストの実行,統合テストスイートであるが,組み込みのテスト機構は汎用性に乏しい.

我々は,テストフレームワークの汎用性を高めるために,構成管理ツール特有の振る舞い(例えば,パッケージのインストールやシステムユーザの作成など)を抽出・一般化し,それらをテストするためのコマンドをOS毎に分離,その上でOS毎のコマンドの違いを吸収するレイヤーを設けることにより,汎用コマンド実行フレームワークを定義した.

続いて,テストコードの記述の抽象度を高め可読性を上げるために,宣言的な記法で汎用コマンド実行フレームワークを操作できる制御テストフレームワークを定義した.これにより,テストコードのメンテナンス性を高め,サーバの運用・管理コストを低減することができる.また,フレームワークを用途別に分離して定義することにより,制御テストフレームワークを独自の記法に変更することも容易である.例えば,本論文で提案するテストフレームワークでは,テスト記法してRSpec\cite{rspec}を採用しているが,minitest\cite{minitest}に差し替えたり,あるいはまったく独自の記法に差し替えたりすることも可能である.このテストフレームワークをserverspec\cite{serverspec}と名付けた.

本論文の構成について述べる.2章ではサーバの構成管理とテスト手法について更に詳しく述べる.3章では提案するサーバのテスト手法を詳細に述べ,4章では提案手法によりもたらされた成果を論じ,5章で結びとする.

\section{サーバの構成管理とテスト手法}

安価なUNIXライクOSを搭載したサーバが普及し,TCP/IPにより異なるOSを搭載したサーバ同士がネットワーク接続可能になったことにより,システムが大規模化・複雑化したことで,シェルスクリプトに代わり,サーバの設定を抽象的なコードで扱う構成管理手法が登場した.\cite{cfengine paper1}


\section{提案するサーバテスト手法}

\subsection{テスト種別の選択}

2章にて,従来のテスト手法では構成管理ツール独立性とOS・ディストリビューション汎用性の双方を満たすものが,どのテスト種別においても存在しない,ということを述べた.そこでまず,本論文で提案する手法を,どのテスト種別に適用するのか選択を行う.

種別(1)の単体テストは既に述べたように,サーバのテスト手法としては不十分である.また,構成管理ツール固有の言語により書かれたコードをテストするものなので,構成管理ツールから切り離して考えることができない.よって単体テストでは構成管理ツール独立性を満たすことは不可能である.ゆえに単体テストは提案手法の対象から除外する.種別(3)の受け入れテストは,サーバの外からの振る舞いをテストするもので,内部状態をテストするものではない.サーバ構成を記述したコードのテストは,コードによってもたらされたサーバ内部の設定状態をテストすべきものと考える.よって受け入れテストは提案手法の対象から除外する.残る種別(2)の結合テストは,サーバの内部状態を網羅的にテスト可能であり,サーバ構成を記述したコードのテストとして相応しいものである.よって(2)の結合テストを提案手法の適用対象とする.

\subsection{要件の考察}

次に,結合テストにおいて構成管理ツール独立性とOS・ディストリビューション汎用性の双方を満たすための要件について考察する.

特定の構成管理ツールからの独立性を満たせない理由は2つある.ひとつは,テストスイートはテストに必要なプロセスをすべて実行するため,テスト用VMを構築する機能も併せ持つが,この機能が特定の構成管理ツールに依存しているためである.もうひとつは,テストコードにOS・ディストリビューション汎用性を持たせるために,テストの実装が構成管理ツールが元々持っている機能に依存しているためである.

OS・ディストリビューション汎用性が満たせないのは,テストがシェルコマンドを直接記述する実装になっており,OS・ディストリビューションの違いをテストコードを書く者自らが意識しないといけないからである.この考察から,提案するテスト手法に必要な要件は以下の通りとなる.

\begin{enumerate}
  \item テストスイートではなくテストのみに特化する
  \item テストの実装を特定の構成管理ツールに依存しない
  \item OS・ディストリビューションの違いをテストコードを書く者に意識させない
\end{enumerate}

\subsection{要件を満たすための手法の提案}

考察した要件を満たすために,運用業務で発生するコマンド群,特に確認作業に必要なコマンド群の体系化・抽象化を行う.そのためにまずはOS・ディストリビューション毎にコマンドを分離し,統一的なAPIでコマンドを呼び出すことができる汎用コマンド実行フレームワークを定義する.次に,汎用コマンド実行フレームワークを宣言的かつ自然言語に近い記法で操作できる制御テストフレームワークを定義する.汎用コマンド実行フレームワークと制御テストフレームワークの仕組みおよびその関係を\figref{fig:framework}に示す.汎用コマンド実行フレームワークではまず,構成管理ツール固有の振る舞い(パッケージインストール等)を抽出する.そして要件(1)を満たすために,振る舞いのテストに特化したAPIを定義する.更にAPIから呼び出されるコマンドをOS・ディストリビューション毎に定義し,要件(2)を満たすために,APIとコマンド群の間にはOS・ディストリビューションを判別して自動で適切なコマンドを返すレイヤーを設ける.ここは構成管理ツール等,特別なソフトウェアを必要としない方式をとる.制御テストフレームワークではまず,要件(3)を満たすため,テストコード記述の抽象度を高め可読性を上げるために,宣言的かつ自然言語に近い記法で汎用コマンド実行フレームワークを操作するための記法の定義を行う.次に記法内の各命令と実際に呼び出す汎用コマンド実行フレームワークのAPIメソッドをひもづける.



\begin{figure}[tb]
  \includegraphics{framework-overview.png}
  \caption{汎用コマンド実行フレームワークと制御テストフレームワークの仕組みと関係}
  \label{fig:framework}
\end{figure}

\subsection{提案手法の実装}

提案手法に基づき実装した汎用コマンド実行フレームワークをspecinfra\cite{specinfra},制御テストフレームワークをserverspec\cite{serverspec}と名付けた.specinfra,serverspecともに実装にはRubyを採用している.Rubyを採用した理由はRSpecを利用するためである.RSpecはテストフレームワークとして実績があり,自然言語に近い形でテストコードを記述することができるという要件を満たし,記法の拡張が可能であることから採用した.

\begin{figure}[tb]
\setbox0\vbox{
\begin{verbatim}
describe file("/etc/password") do
  it { should be_file }
end

describe file("/tmp") do
  it { should be_directory }
end

describe file("/var/run/unicorn.sock") do
  it { should be_socket }
end

describe file("/etc/httpd/conf/httpd.conf") do
  its(:content) do
    should match /ServerName www.example.jp/
  end
end
\end{verbatim}
}
\centerline{\fbox{\box0}}
\caption{serverspecによりファイルをテストするためのコード\label{fig:test-files-with-serverspec}}
\end{figure}

\begin{figure}[tb]
\setbox0\vbox{
\begin{verbatim}
describe user("root") do
  it { should exist }
  it { should have_uid 0 }
  it { should belong_to_group "root" }
  it { should have_home_directory "/root" }
  it { should have_login_shell "/bin/bash" }
end

describe group("root") do
  it { should exist }
  it { should have_gid 0 }
end
\end{verbatim}
}
\centerline{\fbox{\box0}}
\caption{serverspecによりシステムユーザ/グループをテストするためのコード\label{fig:test-users-with-serverspec}}
\end{figure}

RSpecを採用することによりテストコードがどのように書けるのかを例で示す.\figref{fig:test-files-with-serverspec}にserverspecによりファイルに対してテストを行うためのコードを示す.describeではテストの対象となるサーバ上のリソースを指定する.この図ではfile("/etc/passwd")などがそれにあたる.これによりテスト対象が/etc/passwdというファイルであることを指定する.テストしたい内容はit \{ should ... \}といった形で記述する.例えば,it \{ should be\_file \}は,対象リソースがファイルとして存在する,ということをテストするためのコードである.また,its(:content)といった形で指定することで,対象リソースそのものだけではなく,リソースに付随するもの,この例ではファイルの内容についてもテストすることができる.

\figref{fig:test-users-with-serverspec}にシステムユーザ/グループに対してテストを行うためのコードを示す.この例では,rootユーザが存在し,uidが0,rootグループに所属,ホームディレクトリが/root,ログインシェルが/bin/bashであること,rootグループが存在し,gidが0であることをテストしている.

\subsection{serverspecの特徴}

提案手法に基づき実装したserverspecの特徴について述べる.特徴の一つとして,特定の構成管理ツールに依存していないことが挙げられる.そのため,どの構成管理ツールを利用していてもserverspecを利用することができる.それだけにとどまらず,構成管理ツールを利用していない場合でもserverspecを利用することができる.ゆえに特定の構成管理ツール依存のテストツールと比べて利用の間口が広いと言える.また,特定の構成管理ツールに依存していないということは,テスト対象のサーバに特定のソフトウェアを入れる必要がないということでもある.serverspecはテスト対象サーバでsshdが動いてさえいれば,Rubyすら入れる必要がない.そのため特定の構成管理ツール依存のテストツールと比較して利用の敷居が低い.

二つ目の特徴はTest Kitchenやrspec-systemのような統合テストスイートと比較して単機能な点である.単機能であるため他のツールとも組み合わせやすく,同種ツールとしてとりあげたTest Kitchenやrspec-systemには,ツール標準のテスト機構をserverspecで置き換えるためのプラグインや,Vagrant\cite{vagrant}と連携してVMのテストを行うプラグインが存在する.

三つ目の特徴は記法の汎用性と抽象度の高さである.汎用性を高めたため,OS・ディストリビューションの違いを気にすることなくテストを容易に書くことができる.また抽象度が高いためテストコードの可読性が高く,メンテナンス性が高い.

\section{提案手法の評価}

提案手法を実装したserverspecを採用している企業がいくつか見受けられる.また,同種ツールとしてとりあげたTest Kitchenやrspec-systemには,ツール標準のテスト機構をserverspecで置き換えるbusser-serverspecやrspec-system-serverspecが存在する.他にも,Vagrantと連携してVMのテストを行うvagrant-serverspecというツールが存在する.このように,serverspecは単体利用だけではなく,同種ツールの一機能として取り込まれたり,他種ツールと連携する形でも利用が広がっている.

serverspecがなぜ同種のツールと比べて広く使われているのかを考察する.まずひとつは特定の構成管理ツールに依存していないことが挙げられる.そのためどの構成管理ツールを利用していてもserverspecは利用できる.それだけにとどまらず,構成管理ツールを利用していない場合でもserverspecを利用することができる.そのため特定の構成管理ツール依存のテストツールと比べて利用の間口が広い.

特定の構成管理ツールに依存していないということは,テスト対象のサーバに特定の構成管理ツールを入れる必要がないということでもある.serverspecはテスト対象サーバでsshdが動いてさえいれば,Rubyすら入れる必要がない.そのため特定の構成管理ツール依存のテストツールと比較して利用の障壁が低い.

広く使われている理由の二つ目は記法の汎用性と抽象度の高さである.汎用性を高めたため,OS・ディストリビューションの違いを気にすることなくテストを容易に書くことができる.また抽象度が高いためテストコードの可読性が高く,メンテナンス性が高い.例として,Test Kitchenでは標準でbats\cite{bats}を利用してテストを書くが,batsによるUbuntu上でのテストコードを\figref{fig:test-with-bats-on-ubuntu}に示す.また,同じ内容のテストをSolaris向けに書く場合の例を\figref{fig:test-with-bats-on-solaris}に示す.serverspecによるテストコードは,OSが何であっても\figref{fig:test-with-serverspec}で示すようなコードになる.このように,提案手法ではOSの違いを意識することなく,テストコードを記述することができる.

\begin{figure}[tb]
\setbox0\vbox{
\begin{verbatim}
@test "The package apache2 is installed" {
  dpkg-query -f '${Status}' -W apache2 \
    | grep '^install ok installed$'
}

@test "The apache2 service is running" {
  service apache2 status
}

@test "Port 80 is listening" {
  netstat -tunl | grep ":80 "
}
\end{verbatim}
}
\centerline{\fbox{\box0}}
\caption{batsによるUbuntu上でのテストコード\label{fig:test-with-bats-on-ubuntu}}
\end{figure}

\begin{figure}[tb]
\setbox0\vbox{
\begin{verbatim}
@test "The package apache2 is installed" {
  pkg list -H apache2
}
 
@test "The apache2 service is running" {
  svcs -l apache2 | egrep '^status *online$'
}
 
@test "Port 80 is listening" {
  netstat -an | grep LISTEN | grep ".80 "
}
\end{verbatim}
}
\centerline{\fbox{\box0}}
\caption{batsによるSolaris上でのテストコード\label{fig:test-with-bats-on-solaris}}
\end{figure}

\begin{figure}[tb]
\setbox0\vbox{
\begin{verbatim}
describe package("apache2") do
  it { should be_installed }
end
 
describe service("apache2") do
  it { should be_running }
end
 
describe port(80) do
  it { should be_listning }
end
\end{verbatim}
}
\centerline{\fbox{\box0}}
\caption{serverspecによるテストコード\label{fig:test-with-serverspec}}
\end{figure}


\begin{figure}[tb]
\setbox0\vbox{
\begin{verbatim}
@test "/etc/sudoers is not readable by others" {
  ls -l /etc/sudoers | egrep '^......-..'
}
\end{verbatim}
}
\centerline{\fbox{\box0}}
\caption{batsによる/etc/sudoersが他人から読めないことをテストするコード\label{fig:test-permission-with-bats}}
\end{figure}

\begin{figure}[tb]
\setbox0\vbox{
\begin{verbatim}
describe file("/etc/sudoers") do
  it { should_not be_readable.by("others") }
end
\end{verbatim}
}
\centerline{\fbox{\box0}}
\caption{serverspecによる/etc/sudoersが他人から読めないことをテストするコード\label{fig:test-permission-with-serverspec}}
\end{figure}

また別の例として,/etc/sudoersが他人から読めないことをテストするコードの例を示す.batsを利用する場合は\figref{fig:test-permission-with-bats}に示すコードとなる.serverspecを利用する場合は\figref{fig:test-permission-with-serverspec}に示すコードとなる.このように, batsの場合はテストコードだけでは何をテストしているのか判別しにくいため,説明用のテキストが必要となる.一方serverspecはテストコードだけでテスト内容が理解できるため,別途説明用のテキストを必要としない.

構成管理ツール依存を排除し,汎用的なのに加え,単機能でもある.ゆえに他ツールとも組み合わせやすく,従来からあるテストツールや多種ツールに取り込まれる形での利用も広がっている.

弊社ではserverspecを採用することによって,古くからあるPuppetマニフェスト(Puppetの記法で書かれたテストコード)をリファクタリングしようという動きが現場エンジニアの間で活発になっている.

とは言え,利用の広がりについては,きちんとした調査の結果ではなく,定量的な評価は今後の課題である.

\section{まとめ}

本論文では,Test-Driven Infrastructureを支援するための従来テストフレームワーク手法の問題点について指摘し,解決するために必要な要件として,構成管理ツール独立性とOS・ディストリビューション汎用性を提案した.また,これらの要件を満たすために,汎用コマンド実行フレームワークと制御テストフレームワークに分離して定義する手法についても提案した.これらふたつの実装例としてspecinfraとserverspecを紹介した.

serverspecの登場によってTest-Driven Infrastuctureというプロセスが国内でも徐々に認知され,この分野の今後の発展が期待される.またspecinfraを基盤とした,serverspecよりも優れたテストフレームワーク実装の登場や,確認作業以外の運用業務に必要なコマンド群を体系化することによるテスト以外へのspecinfraの応用,例えば構成管理ツールへの応用なども今後期待される.

今後の課題として,提案手法による運用効率向上の定量的評価ができていないので,評価手法について検討するとともに,手法に関する助言を募りたい.



\begin{acknowledgment}
serverspec実装にあたり、提案手法の原型となる実装を示してくれた株式会社paperboy\&co.伊藤洋也氏に感謝する。
\end{acknowledgment}

\bibliographystyle{ipsjunsrt}
\bibliography{serverspec}

\end{document}
